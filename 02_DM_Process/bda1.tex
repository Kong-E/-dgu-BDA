% Options for packages loaded elsewhere
\PassOptionsToPackage{unicode}{hyperref}
\PassOptionsToPackage{hyphens}{url}
%
\documentclass[
]{article}
\usepackage{amsmath,amssymb}
\usepackage{lmodern}
\usepackage{iftex}
\ifPDFTeX
  \usepackage[T1]{fontenc}
  \usepackage[utf8]{inputenc}
  \usepackage{textcomp} % provide euro and other symbols
\else % if luatex or xetex
  \usepackage{unicode-math}
  \defaultfontfeatures{Scale=MatchLowercase}
  \defaultfontfeatures[\rmfamily]{Ligatures=TeX,Scale=1}
\fi
% Use upquote if available, for straight quotes in verbatim environments
\IfFileExists{upquote.sty}{\usepackage{upquote}}{}
\IfFileExists{microtype.sty}{% use microtype if available
  \usepackage[]{microtype}
  \UseMicrotypeSet[protrusion]{basicmath} % disable protrusion for tt fonts
}{}
\makeatletter
\@ifundefined{KOMAClassName}{% if non-KOMA class
  \IfFileExists{parskip.sty}{%
    \usepackage{parskip}
  }{% else
    \setlength{\parindent}{0pt}
    \setlength{\parskip}{6pt plus 2pt minus 1pt}}
}{% if KOMA class
  \KOMAoptions{parskip=half}}
\makeatother
\usepackage{xcolor}
\usepackage[margin=1in]{geometry}
\usepackage{color}
\usepackage{fancyvrb}
\newcommand{\VerbBar}{|}
\newcommand{\VERB}{\Verb[commandchars=\\\{\}]}
\DefineVerbatimEnvironment{Highlighting}{Verbatim}{commandchars=\\\{\}}
% Add ',fontsize=\small' for more characters per line
\usepackage{framed}
\definecolor{shadecolor}{RGB}{248,248,248}
\newenvironment{Shaded}{\begin{snugshade}}{\end{snugshade}}
\newcommand{\AlertTok}[1]{\textcolor[rgb]{0.94,0.16,0.16}{#1}}
\newcommand{\AnnotationTok}[1]{\textcolor[rgb]{0.56,0.35,0.01}{\textbf{\textit{#1}}}}
\newcommand{\AttributeTok}[1]{\textcolor[rgb]{0.77,0.63,0.00}{#1}}
\newcommand{\BaseNTok}[1]{\textcolor[rgb]{0.00,0.00,0.81}{#1}}
\newcommand{\BuiltInTok}[1]{#1}
\newcommand{\CharTok}[1]{\textcolor[rgb]{0.31,0.60,0.02}{#1}}
\newcommand{\CommentTok}[1]{\textcolor[rgb]{0.56,0.35,0.01}{\textit{#1}}}
\newcommand{\CommentVarTok}[1]{\textcolor[rgb]{0.56,0.35,0.01}{\textbf{\textit{#1}}}}
\newcommand{\ConstantTok}[1]{\textcolor[rgb]{0.00,0.00,0.00}{#1}}
\newcommand{\ControlFlowTok}[1]{\textcolor[rgb]{0.13,0.29,0.53}{\textbf{#1}}}
\newcommand{\DataTypeTok}[1]{\textcolor[rgb]{0.13,0.29,0.53}{#1}}
\newcommand{\DecValTok}[1]{\textcolor[rgb]{0.00,0.00,0.81}{#1}}
\newcommand{\DocumentationTok}[1]{\textcolor[rgb]{0.56,0.35,0.01}{\textbf{\textit{#1}}}}
\newcommand{\ErrorTok}[1]{\textcolor[rgb]{0.64,0.00,0.00}{\textbf{#1}}}
\newcommand{\ExtensionTok}[1]{#1}
\newcommand{\FloatTok}[1]{\textcolor[rgb]{0.00,0.00,0.81}{#1}}
\newcommand{\FunctionTok}[1]{\textcolor[rgb]{0.00,0.00,0.00}{#1}}
\newcommand{\ImportTok}[1]{#1}
\newcommand{\InformationTok}[1]{\textcolor[rgb]{0.56,0.35,0.01}{\textbf{\textit{#1}}}}
\newcommand{\KeywordTok}[1]{\textcolor[rgb]{0.13,0.29,0.53}{\textbf{#1}}}
\newcommand{\NormalTok}[1]{#1}
\newcommand{\OperatorTok}[1]{\textcolor[rgb]{0.81,0.36,0.00}{\textbf{#1}}}
\newcommand{\OtherTok}[1]{\textcolor[rgb]{0.56,0.35,0.01}{#1}}
\newcommand{\PreprocessorTok}[1]{\textcolor[rgb]{0.56,0.35,0.01}{\textit{#1}}}
\newcommand{\RegionMarkerTok}[1]{#1}
\newcommand{\SpecialCharTok}[1]{\textcolor[rgb]{0.00,0.00,0.00}{#1}}
\newcommand{\SpecialStringTok}[1]{\textcolor[rgb]{0.31,0.60,0.02}{#1}}
\newcommand{\StringTok}[1]{\textcolor[rgb]{0.31,0.60,0.02}{#1}}
\newcommand{\VariableTok}[1]{\textcolor[rgb]{0.00,0.00,0.00}{#1}}
\newcommand{\VerbatimStringTok}[1]{\textcolor[rgb]{0.31,0.60,0.02}{#1}}
\newcommand{\WarningTok}[1]{\textcolor[rgb]{0.56,0.35,0.01}{\textbf{\textit{#1}}}}
\usepackage{graphicx}
\makeatletter
\def\maxwidth{\ifdim\Gin@nat@width>\linewidth\linewidth\else\Gin@nat@width\fi}
\def\maxheight{\ifdim\Gin@nat@height>\textheight\textheight\else\Gin@nat@height\fi}
\makeatother
% Scale images if necessary, so that they will not overflow the page
% margins by default, and it is still possible to overwrite the defaults
% using explicit options in \includegraphics[width, height, ...]{}
\setkeys{Gin}{width=\maxwidth,height=\maxheight,keepaspectratio}
% Set default figure placement to htbp
\makeatletter
\def\fps@figure{htbp}
\makeatother
\setlength{\emergencystretch}{3em} % prevent overfull lines
\providecommand{\tightlist}{%
  \setlength{\itemsep}{0pt}\setlength{\parskip}{0pt}}
\setcounter{secnumdepth}{-\maxdimen} % remove section numbering
\ifLuaTeX
  \usepackage{selnolig}  % disable illegal ligatures
\fi
\IfFileExists{bookmark.sty}{\usepackage{bookmark}}{\usepackage{hyperref}}
\IfFileExists{xurl.sty}{\usepackage{xurl}}{} % add URL line breaks if available
\urlstyle{same} % disable monospaced font for URLs
\hypersetup{
  pdftitle={HW\_PCA},
  pdfauthor={KONG SO YEON},
  hidelinks,
  pdfcreator={LaTeX via pandoc}}

\title{HW\_PCA}
\author{KONG SO YEON}
\date{2022-09-24}

\begin{document}
\maketitle

\hypertarget{q1-load-decathlon2-dataset-and-create-a-new-dataset-excluding-the-rank-and-competition-variables.}{%
\subsection{Q1: Load ``decathlon2'' dataset and create a new dataset
excluding the ``Rank'' and ``Competition''
variables.}\label{q1-load-decathlon2-dataset-and-create-a-new-dataset-excluding-the-rank-and-competition-variables.}}

\begin{Shaded}
\begin{Highlighting}[]
\FunctionTok{library}\NormalTok{(factoextra)}
\end{Highlighting}
\end{Shaded}

\begin{verbatim}
## 필요한 패키지를 로딩중입니다: ggplot2
\end{verbatim}

\begin{verbatim}
## Welcome! Want to learn more? See two factoextra-related books at https://goo.gl/ve3WBa
\end{verbatim}

\begin{Shaded}
\begin{Highlighting}[]
\FunctionTok{data}\NormalTok{(}\StringTok{"decathlon2"}\NormalTok{)}
\FunctionTok{View}\NormalTok{(decathlon2)}

\NormalTok{data }\OtherTok{\textless{}{-}}\NormalTok{ decathlon2[,}\SpecialCharTok{{-}}\FunctionTok{c}\NormalTok{(}\DecValTok{11}\NormalTok{, }\DecValTok{13}\NormalTok{)]}
\end{Highlighting}
\end{Shaded}

\hypertarget{q2-use-the-points-variable-as-the-dependent-variable-and-create-the-independent-variablex-and-dependent}{%
\subsection{Q2: Use the ``Points'' variable as the dependent variable
and create the independent variable(x) and
dependent}\label{q2-use-the-points-variable-as-the-dependent-variable-and-create-the-independent-variablex-and-dependent}}

\begin{Shaded}
\begin{Highlighting}[]
\NormalTok{y }\OtherTok{\textless{}{-}} \FunctionTok{data.frame}\NormalTok{(data[,}\FunctionTok{c}\NormalTok{(}\DecValTok{11}\NormalTok{)])}
\NormalTok{x }\OtherTok{\textless{}{-}}\NormalTok{ data[,}\FunctionTok{c}\NormalTok{(}\DecValTok{1}\SpecialCharTok{:}\DecValTok{10}\NormalTok{)]}
\end{Highlighting}
\end{Shaded}

\hypertarget{q3-conduct-a-principal-component-analysis-using-independent-variable-set-and-check-the-importance-of-components.}{%
\subsection{Q3: Conduct a principal component analysis using independent
variable set and check the importance of
components.}\label{q3-conduct-a-principal-component-analysis-using-independent-variable-set-and-check-the-importance-of-components.}}

\begin{Shaded}
\begin{Highlighting}[]
\NormalTok{pcs }\OtherTok{\textless{}{-}} \FunctionTok{prcomp}\NormalTok{(}\FunctionTok{na.omit}\NormalTok{(x), }\AttributeTok{scale. =}\NormalTok{ T)}
\FunctionTok{summary}\NormalTok{(pcs)}
\end{Highlighting}
\end{Shaded}

\begin{verbatim}
## Importance of components:
##                          PC1    PC2    PC3    PC4     PC5     PC6     PC7
## Standard deviation     1.936 1.3210 1.2320 1.0160 0.78603 0.65444 0.57089
## Proportion of Variance 0.375 0.1745 0.1518 0.1032 0.06178 0.04283 0.03259
## Cumulative Proportion  0.375 0.5495 0.7013 0.8045 0.86630 0.90913 0.94172
##                            PC8     PC9    PC10
## Standard deviation     0.52857 0.43716 0.33511
## Proportion of Variance 0.02794 0.01911 0.01123
## Cumulative Proportion  0.96966 0.98877 1.00000
\end{verbatim}

\begin{Shaded}
\begin{Highlighting}[]
\NormalTok{pcs}\SpecialCharTok{$}\NormalTok{rotation}
\end{Highlighting}
\end{Shaded}

\begin{verbatim}
##                      PC1        PC2          PC3         PC4        PC5
## X100m        -0.42290657  0.2594748 -0.081870461  0.09974877 -0.2796419
## Long.jump     0.39189495 -0.2887806  0.005082180 -0.18250903  0.3355025
## Shot.put      0.36926619  0.2135552 -0.384621732  0.03553644 -0.3544877
## High.jump     0.31422571  0.4627797 -0.003738604  0.07012348  0.3824125
## X400m        -0.33248297  0.1123521 -0.418635317  0.26554389  0.2534755
## X110m.hurdle -0.36995919  0.2252392 -0.338027983 -0.15726889  0.2048540
## Discus        0.37020078  0.1547241 -0.219417086  0.39137188 -0.4319091
## Pole.vault   -0.11433982 -0.5583051 -0.327177839 -0.24759476 -0.3340758
## Javeline      0.18341259  0.0745854 -0.564474643 -0.47792535  0.1697426
## X1500m        0.03599937 -0.4300522 -0.286328973  0.64220377  0.3227349
##                      PC6         PC7         PC8         PC9        PC10
## X100m         0.16023494 -0.03227949  0.35266427 -0.71190625  0.03272397
## Long.jump     0.07384658  0.24902853  0.72986071 -0.12801382  0.02395904
## Shot.put      0.32207320  0.23059438 -0.01767069  0.07184807 -0.61708920
## High.jump     0.52738027  0.03992994 -0.25003572 -0.14583529  0.41523052
## X400m        -0.23884715  0.69014364 -0.01543618  0.13706918  0.12016951
## X110m.hurdle  0.26249611 -0.42797378  0.36415520  0.49550598 -0.03514180
## Discus       -0.28217086 -0.18416631  0.26865454  0.18621144  0.48037792
## Pole.vault    0.43606610  0.12654370 -0.16086549  0.02983660  0.40290423
## Javeline     -0.42368592 -0.23324548 -0.19922452 -0.33300936  0.02100398
## X1500m        0.10850981 -0.34406521 -0.09752169 -0.19899138 -0.18954698
\end{verbatim}

\hypertarget{q4-choose-some-components-to-conduct-a-regression-analysis-to-predict-the-dependent-variable.-how-many-components-did-you-choose-explain.}{%
\subsection{Q4: Choose some components to conduct a regression analysis
to predict the dependent variable. How many components did you choose?
Explain.}\label{q4-choose-some-components-to-conduct-a-regression-analysis-to-predict-the-dependent-variable.-how-many-components-did-you-choose-explain.}}

PC1, PC2, PC3, PC4, PC5, PC6을 선택했다. PC6에서 Cumulative Proportion가
약 91\%이기 때문에 종속변수 y를 설명하기에 충분한 비율이다.

\end{document}
